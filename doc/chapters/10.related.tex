\section{Related Work}

\paragraph{RP Debugging} 
REScala~\cite{salvaneschi2014rescala} is a RP library for Scala, based on Scala.React. 
Recently a debugger model was created for REScala, called ``RP Debugging''~\cite{salvaneschi2016debugging}, 
featuring a dependency graph visualization, breakpoints, a query language and performance monitoring.
The debugger fully integrates with the Eclipse IDE and the Scala debugger facilities, 
creating a (Scala) developer experience and a feature RxFiddle currently can not offer: reactive breakpoints. 
However breakpoints are arguably not as useful as ``simulated breakpoints'' (Section \ref{breakpoints}, \nameref{breakpoints}).
Furthermore, our debugger design supports multiple languages, and works outside of the IDE,
in the browser environment and/or connecting to a production system.
Rx has different reactive semantics and a more powerful, but also more extensive API, which includes operators acting in the time domain (\code{delay}, etc.).
Therefore, we argue that seeing the many values in a flow over time is very valuable; 
RP Debugging shows the latest values at the selected time.

\paragraph{RP Visualization} 
RxMarbles~\cite{rxmarbles} visualizes single Rx operators, for the purpose of learning and comprehension.
Users can modify the diagrams by dragging the events and instantly see the changes reflected in the output.
By using specific precoded inputs and timings the essence of the operator is made clear.
In the soon to be released RxViz~\cite{rxviz}, 
Moroshko takes a similar approach, but uses code instead of prepared inputs.
Where RxMarbles is limited to non higher order flows, RxViz subscribes to all inner Observables when it detects a higher order Observable showing them concurrently.
In contrast to our work, these tools focus only on teaching the behavior of single operators.

\paragraph{Omniscient Debugging} Omniscient debuggers~\cite{pothier2009back} are ``all-knowing debuggers'', 
that trace, store and query all events in a program execution.
When storing vasts amount of program execution information, performance and efficiency becomes very much a problem and research in omniscient debuggers focus on this specifically.
We also trace events of the entire execution, however in contrast to omniscient debuggers we only store trace events regarding RP data flows.
The RP semantics allow us to create future optimizations, for example retaining only the active flow structure, while the flow's data is kept in a rolling buffer.

\paragraph{Dynamic Analysis.}
The study of program execution is called ``dynamic analysis''~\cite{cornelissen2009systematic}.
In most cases dynamic analysis involves a \textit{post mortem} analysis, 
where first the program is run, collecting an execution trace, and then the trace data is analyzed to create a visualization.
The various derived visualizations, like class and instance interaction graphs, function invocation histories~\cite{lange1995program}, invocation views and sequence diagrams~\cite{cornelissen2008execution} show the possibility to use trace information for debugging.
Arguably an on-line analysis is more useful for debugging than the standard post mortem analysis. Reiss, in reference~\cite{reiss2006visualizing}, mentions the compromises that have to be made to make an on-line analysis:
reduced tracing is required to not slow down the system (known as the observer-effect) and
fast analysis is required to lower the cost of getting to the visualization, to not discourage the users.
In our design, we handle the same compromises as they are relevant for RP debugging too, and our JavaScript trace implementation bears resemblance to that of Program Visualiser~\cite{lange1995program}.

% \todo{
% \begin{enumerate}
%  \item Cornelissen, TU Delft, survey on Program Compr. through Dynamic Analysis
%  		lists limitations: 
% 		   - incompleteness, only part of domain
% 			 - which scenarios to analyse
% 			 - scalability (wrt human cognitive load) 
% 		 	 - observer effect (multi-threaded, realtime) changes execution
% \end{enumerate}
% }

% \todo{
% \begin{enumerate}
% 	\item Atlas (2014): rich graph representation, can be used to build call graphs, dataflow graphs, dependency graphs
% %	\item RP debugging (2016, Guido): REScala, data flow graph, breakpoints
% %	\item BIGDEBUG Spark debugging (2017): data flow graph, non-pausing simulated breakpoints, data provenance
% \end{enumerate}
% }

% \textbf{Tracing \& Automated visualization for comprehension}
% \todo{
% \begin{enumerate}
%  \item Lange 1995, Program Visualizer for C++
%  \item story flow: visualize stories over time, comprehending relations, interactive visualization
%  \item Weck \& Tichy, Visualizing Data-Flows in Functional Programs
%  \item Srinivasan, ICPC16, ``Case Studies of Optimized Sequence Diagram for Program Comprehension", Texas A\&M Univerity
% %  \item Misha Moroshko (Facebook) Rx visualization
% \end{enumerate}
% }

\paragraph{Understanding Debugging}
Debugging for general purpose languages revolves around 
attaching a debugger,
stepping through the code,
attaching code or data breakpoints,
navigating along different calls in the call stack and 
examining variables and results of expressions~\cite{Spinellis2017}.
However, existing research, measuring how these different tasks are part of the developers work day, found that
while developers spend much time on comprehending code, they do not spend much time inside the IDE's debugger~\cite{minelli2015know}.
Beller et al.~\cite{beller2017behavior} found that only 23\% of their subjects actively use the IDE's debugger,
with the most common action being adding breakpoints, followed by stepping through code.
The automated tooling of these studies did not measure different kinds of debugging other than using the IDE provided tools, 
however Beller's survey indicates that 71\% also uses \code{printf} statements for debugging.
No indication was given of any RP language and libraries used by the subjects in the study, 
but the observation that \printfdebugging{} is common, matches our experience with debugging reactive programs.

% What comprises debugging?
% \item Maybe Zeller / Spinellis?

% \item Petrillo, ``Towards Understanding..." e.g. Swarm debugging
% \item Minelli (know what you did last summer)
% \item Moritz (watchdog 2.0)

\paragraph{Debugging for Program Comprehension}
Both debugging and comprehension are processes in the work of programmers.
Initially, comprehension was seen as a distinct step programmers had to make
prior to being able to debug programs~\cite{katz1987debugging}, 
but this distinction is criticized by Gilmore, saying we must view
``debugging as a design activity''~\cite{gilmore1991models}, 
part of creating and comprehending programs. 
Maalej et al.~\cite{Maalej2014} interviewed professional developers 
and found that developers require runtime information to understand a program,
and that debugging is frequently used to gather this runtime information.
This supports our view that debugging is not only used for fault localization,
but also for comprehension.

% \item Katz, distinct step
% \item Gilmore, comprehension + debugging == linked
% \item Maalej: professionals, avoid deep comprehension, sharing knowlegde
