\section{Conclusions} Observing the current debugging practices, this
work shows the prevalent method for RP debugging is \printfdebugging{}.
To provide a better alternative, we have created a RP debugger design
and presented the RxFiddle implementation, a RP debugger for RxJS, which
allows developers to:  (1) gain a high-level overview of the reactive
data flow structure and dependencies, (2) investigate the values and
life-cycle of a specific data flow, at run-time.

Through an experiment we show that RxFiddle is an viable alternative for
traditional debugging and in some cases outperforms traditional
debugging in terms of time spent.  There are several promising
directions for improving our design.  Specifically scalability could be
improved and different edge visualizations could be explored, to improve
the usability of the tool.  Furthermore, by leveraging already captured
meta data about timing of events, even more insight could be provided.
At the implementation level, we plan to extend RxFiddle to other members
of the Rx-family of languages.

In this paper, we make the following concrete contributions:
\begin{itemize}
    \item[(1)]
        Design of a RP debugger
    \item[(2)]
        The implementation of the debugger for RxJS, and the service
        RxFiddle.net, a platform for the debugger in an online
        environment with code sharing functionality.
\end{itemize}

The debugger and the platform are open source and are available online
at~\cite{rxfiddle-doi}.
