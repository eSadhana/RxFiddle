\section{Research Design} % page 212 of Creswell-Cap-10.pdf
To answer our research questions, we employ a three-phase Sequential
Exploratory Strategy, one of the mixed methods research approaches~\cite
{creswell2013research,hanson2005mixed}, as shown in Table~%
\ref{research-methods}.  First, we interview professional developers and
review available documentation (RQ1) to form a understanding about
current debugging practices, second we apply this understanding to
design a debugger and implement it to test its feasibility (RQ2),
finally we validate the debugger using an experiment (RQ3).

\begin{table}[t]
    \centering
    \begin{tabularx}{\columnwidth}
        {lllX}
        \hline
        \textbf{} & \textbf{Method} & \textbf{Focus} \\
        \hline
        \multirow{2}{*}{RQ1} & Interview & What are current practices \\
        & Literature & What is recommended \\
        \multirow{2}{*}{RQ2} & Design & What can a RP debugger show \\
        & Implement & Extract meta information from Rx \\
        RQ3 & Experiment & Quantification of effect on debugging \\
        \hline
    \end{tabularx}
    \caption{Research Methods used in the study}%
    \label{research-methods}
\end{table}
