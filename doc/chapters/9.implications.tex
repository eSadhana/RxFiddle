\subsection{Implications}
The developers using Rx in practice now have an alternative to \printfdebugging{}. 
We recommend developers to try RxFiddle on their codebase to better understand the reactive behavior of their applicatoin, 
and potentially detect and verify (performance) bugs they are not aware of.
At least one example of this has already occurred in practice:
one of our interview subjects reported a bug\footnote{\url{https://github.com/ReactiveX/rxjs/issues/2661}} in the \code{groupBy} implementation of RxJS,
which resulted in retention of subscriptions, increased memory usage and finally an out-of-memory exception;
the subject detected the bug in practice and required extensive amount of debugging involving the NodeJS debugger to trace down,
but could be validated quickly when examining the life-cycle events in RxFiddle.

Contributors of RP libraries should use tools like the RxFiddle visualizer in documentation to provide executable samples,
which would allow for a better learning experience, 
and at the same time introduces novice developers to other ways of debugging than \printfdebugging{}.
