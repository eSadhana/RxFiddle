\subsection{Limitations}

\paragraph{Multiple inputs and outputs} If we compare our debugger
visualization to the visualization of learning tools, like RxMarbles~\cite
{rxmarbles} or the soon to be released RxViz of Misha Moroshko~\cite{rxviz},
the main difference is that those tools show all inputs and outputs of a
single operator concurrently, for comprehension of that one operator,
while RxFiddle shows one input and one output per Marble Diagram of a
single full flow through the whole graph.  The choice to show a full
flow allows developers to trace events from the start till end of the
flow, but restricts us in showing only a single ancestor flow per node
at each vertical position, as adding a third dimension would clutter the
(currently 2D) visualization.  For future research it would be
interesting to compare (1) the different ways Observable streams can be
combined in Marble Diagrams and (2) which visualization elements can be
added to explicitly show causality and lineage for events and show
durations for subscriptions.

\paragraph{Edge visualization} In our graph visualization, the edges
represent the dependencies and the path of the events.  Nodes with
multiple incoming edges \emph{merge} the events, however users could
falsely think that all event data ends up in the outgoing path:  besides
for \textit{merging data} Rx also uses Observables \textit{for timing},
as durations (\code{window}), as stop conditions (\code{takeUntil}) and
as toggles (\code{pausable}).  Different visual representations for
joining paths could be explored to distinguish between the use of an
Observable for data or for timing.

\paragraph{Graph scalability} Debugging large reactive systems over
longer periods of time can result in significantly larger Observable
graphs and Marble Diagrams than is currently evaluated.  During tests of
RxFiddle with larger applications like RxFiddle itself and an existing
Angular application the graph became too large to render in a reasonable
amount of time.  Besides rendering performance, a potentially even
bigger issue is with communicating large graphs to the developer.  We
propose several extensions to RxFiddle to remedy this issue:  (1)
pruning the graph of old flows to show only the active flows, (2)
bundling flows that have the same structure and only rendering a single
instance offering a picker into the flow of interest, (3) collapsing
certain parts of the graph that are local to one source file or
function, (4) adding search functionality to quickly identify flows by
operator or data values, (5) support navigation between code \& graph.

\paragraph{Marble Diagram scalability} Furthermore we think that while
Marble Diagrams are useful for small to medium amount of events ($ < 20 $),
both better performance and better functionality would be achieved by
providing a different interface for high volume flows.  Above a certain
threshold of events this high volume interface could be the default,
offering features like (1) filtering, (2) watch expressions (to look
deeper into the event's value), and advanced features like (3)
histograms \& (4) Fast Fourier Transform (FFT) views.  Manually
examining these distinct events could take a long time.  In contrast, a
debugger could leverage the run-time information about the events that
actually occur, to provide a UI.  Advanced features like histograms
could help the filtering process, while FFT could offer new ways to
optimize the application by doing smarter windowing, buffering and
sampling later on in the chain.

\paragraph{Breakpoints}%
\label{breakpoints} Placing traditional breakpoints in a reactive
program stops the system from being reactive, and therefore can change
the behavior of the system.  This was our reason not to include
breakpoints in RxFiddle.  However, the behavior of breakpoints is
twofold:  they allow us to modify the application state by interacting
with the variables in scope, but they also provide a way to be notified
of an event occurrence.  While the first is arguably not desirable for
reactive systems, the notification property might be a good addition to
RxFiddle.  BIGDEBUG~\cite{Gulzar2016}, a debugging solution for systems
like Spark~\cite{zaharia2012resilient}, introduces \textit{simulated
breakpoints} for this purpose.  When a simulated breakpoint is reached,
the execution resumes immediately and the required lineage information
of the breakpoint is collected in a new independent process.
Implementing this for RxFiddle is a matter of creating the right UI as
the required lineage data is already available.
