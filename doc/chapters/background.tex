\section{Background}
\label{background}
In this section we give an overview of related work 
and the context of this research.

\textbf{Debugging for Program Comprehension.}
Both debugging and comprehension are processes in the work of programmers.
Initially comprehension was seen as a distinct step programmers had to make
prior to being able to debug programs~\cite{katz1987debugging}, 
but this distinction is criticized by Gilmore saying we must view 
``debugging as a design activity''~\cite{gilmore1991models}, 
part of creating and comprehending programs. 
Maalej et al.~\cite{Maalej2014} interviewed professional developers 
and found that developers require runtime information to understand a program,
and that debugging is frequently used to gather this runtime information.
This supports our view that `debugging' is not only used for fault localisation,
but also for comprehension.

\todo{
\begin{enumerate}
 \item Gilmore, comprehension + debugging == linked
 \item Maalej: professionals, avoid deep comprehension, sharing knowlegde
 \item Maybe Zeller / Spinellis?
\end{enumerate}
}

\textbf{Dynamic Analysis.}
Although interactive debuggers are commonly used for comprehension 
of the runtime behavior of programs, more specialized tools exist: 
to study a programs execution is called `dynamic analysis' which has 
received substantial attention in the research community,
as surveyed by Cornellissen et al.~\cite{cornelissen2009systematic}.
They categorise on different facets being the 
\textit{activity} [goal of analysis],
\textit{target} [kind of inspected program or system],
\textit{method} [visualization, metrics, online, querying, etc.],
\textit{evaluation} [preliminary, case study, quantitative, etc.].
Most papers apply a \textit{post mortem} analysis, where first the program is run and then the trace data is analyzed to create a visualization.
Reis et al. mention the compromises
that have to be made to make an online analysis: 
reduced tracing is required to not slow down the 
system (known as the observer-effect), fast analysis 
and visualization is required to lower the cost of getting 
to the visualization, to not discourage the users.

Only X of the surveyed papers apply online analysis, rest post mortem.

\todo{
\begin{enumerate}
 \item Cornelissen, TU Delft, survey on Program Compr. through Dynamic Analysis
 		lists limitations: 
		   - incompleteness, only part of domain
			 - which scenarios to analyse
			 - scalability (wrt human cognitive load) 
		 	 - observer effect (multi-threaded, realtime) changes execution
\end{enumerate}
}

\textbf{Measuring Debugging.}
\todo{
\begin{enumerate}
	\item Minelli (know what you did last summer)
	\item Petrillo, ``Towards Understanding..." e.g. Swarm debugging
  \item Moritz (watchdog 2.0)
\end{enumerate}
}

\textbf{Debugging for specific fields}
Most research into debugging focusses on procedural and 
imperative languages~\cite{cornelissen2009systematic}.
Other topics of interest are multi-threading and distributed systems.
Only few focus on other styles, like 
declarative programming~\cite{nilsson1998declarative}.
Debugging in reactive programming~\cite{
	salvaneschi2014empirical,salvaneschi2016debugging}.
	
\todo{
\begin{enumerate}
	\item Atlas (2014): rich graph representation, can be used to build call graphs, dataflow graphs, dependency graphs
	\item RP debugging (2016, Guido): REScala, data flow graph, breakpoints
	\item BIGDEBUG Spark debugging (2017): data flow graph, non-pausing simulated breakpoints, data provenance
\end{enumerate}
}

\textbf{Tracing \& Automated visualization for comprehension}
\todo{
\begin{enumerate}
 \item Lange 1995, Program Visualizer for C++
 \item story flow: visualize stories over time, comprehending relations, interactive visualization
 \item Weck \& Tichy, Visualizing Data-Flows in Functional Programs
 \item Srinivasan, ICPC16, ``Case Studies of Optimized Sequence Diagram for Program Comprehension", Texas A\&M Univerity
 \item Misha Moroshko (Facebook) Rx visualization
\end{enumerate}
}

\subsection{Rx in a nutshell}
To understand how we create the visualization a minimal understanding of RP and the chosen implementation is required. Many RP implementations share a notion of a \textit{Observable}, which is a collection which abstracts over \textit{time}, in contrast to \textit{space} like standard collections.

Figure \ref{sample1} shows a very basic example of a in situ data flow in Rx. First an \textit{Observable} is created, here using the static \code{from} method, then dependent Observables are created using the \code{map} and \code{filter} methods on the Observable instance. Finally we \code{subscribe} to start the data flow and send the data in the flow to the console (eg. JavaScript's stdout).

\begin{figure}
\inputminted[tabsize=2]{javascript}{listings/sample1.js}	
\caption{Creation and transform of Observables}
\label{sample1}
\end{figure}

It is important to note that the Observable is lazy, it is the blueprint of a data flow. Only when you \code{subscribe} to an Observable the data flow is created by recursively subscribing up the stream. \textit{Observer}s are subscribed to each Observable until the source Observable is reached.
This is illustrated in figure \ref{dualgraphs}.
The origin of this design is the duality between Observables and \textit{Iterables}~\cite{meijer2010subject}, where Observers are dual to \textit{Iterators}.

Creating the Observable we will call the \textit{assembly} phase, the phase where the subscribe happens the \textit{subscription} phase and data flows in the \textit{runtime} phase. The three phases can be interleaved for different streams, for example when dealing with higher order Observables,  meaning one could use Observables as values inside the data flow. The Observables used as values have yet to start the second phase while the outer stream is in the runtime phase.

\begin{figure}
\begin{verbatim}
TODO nice figure with Graph & 2 lines (1 up, 1 down)
^ from(1, 2, 3)   			
^ .map(x => x * 2)       v
^ .filter(x => x > 2)    v
  .subscribe()           v
\end{verbatim}
\caption{Observable \& Observer dependencies}
\label{dualgraphs}
\end{figure}
