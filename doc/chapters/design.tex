\section{Research Design}
To answer our research questions we employ a mixed methods research approach as shown in Table~\ref{research-methods}.
First we interview professional developers and review available documentation (RQ1) to form a understanding about current debugging practices in Section \ref{section-practices},
then we apply this understanding to design a debugger and implement it to test its feasibility (RQ2) in Section \ref{section-design},
finally we validate the debugger using an experiment (RQ3) in Section \ref{section-evaluation}.

\begin{table*}[b]
\centering
\caption{Research Methods used in the study}
\label{research-methods}
\begin{tabular}{llll}
\hline
\textbf{}            & \textbf{Method} & \textbf{Type} & \textbf{Focus}                               \\ \hline
\multirow{2}{*}{RQ1} & Interview       & Qualitative   & What are current practices                   \\ \cline{2-4} 
                     & Literature      & Qualitative   & What is recommended                          \\ \hline
\multirow{2}{*}{RQ2} & Design          & Model         & What can a RP debugger show                  \\ \cline{2-4} 
                     & Implement       & Tool          & Collect the required meta information        \\ \hline
RQ3                  & Experiment      & Quantitative  & Quantification of effect on debug time       \\ \hline
\end{tabular}
\end{table*}

\iffalse
\todo{
qualitive data
quantitive
design, implemented, tested with real developers
buzz words: mixed methods, grounded theory
}
\fi