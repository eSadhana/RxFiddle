\section{Debugger Design}
In this section we describe the design of a visualizer for the ReactiveX (Rx) family of RP libraries to answer RQ2.

{\color{red}
\begin{verbatim}
visualization:
- 1. dependency graph
- 2. marble diagram

  motivation:
  - structure defined in code can be scattered
  - see high level structure (helps with composing)
  - see missing dependencies
  - see duplicate flow subscriptions (hurts performance)
  - zoom in to see internals of built in advanced op's
  - is basically automatic documentation

architecture of 3 components:
- 1. host instrumentation
- 2. analyzer: graphs from instrumentation output
- 3. visualizer

  motivation
  - independent instrumentation per platform
  - visualizer separated, so supports both 
    online (live) as offline (post mortem) viz
\end{verbatim}
}


\subsection{Observable \& Observer Graph}
Two dependency graphs can be created, one for the Observables and one for the Observers. In the most basic situations the graphs are equal in shape: there is a mapping between Observers and their Observable. However, with higher order Observables a new type of edge is created in the Observer graph: the 

 The code relates to the setup phase, while the implementation of Rx leads to

RP in general and Rx in particular allow for data flows to be composited into new data flows. 


One can see that every composition from one or more input flows to one or more output flows can be visualized in the form of a graph, very node representing 
The visualizer focusses 

\subsection{Marble Diagrams}

\subsection{Implementation}
{\color{red}
\begin{verbatim}
5 stages:
- 1. instrumentation
- 2. linking / referencing / building a graph
- 3. sending the graph over from the client to the viz
- 4. building the graph on the other side
- 5. layout

explain independency of stages 3-5 on language & platform: 
can run with any Rx if stage 1 & 2 are implemented. 
Efforts done with RxJS 4, RxJS 5 and some limited 
work in RxJava.
\end{verbatim}
}

validation with test set of operators, to be discussed either here or at evaluation.

