\section{Debugger Design}
In this section we describe the design of a visualizer for the ReactiveX (Rx) family of RP libraries to answer RQ2. We propose a visualizer consisting of two parts: a data flow graph and a marble diagram. The data flow graph provides overview, showing how different flows are created and used, while the marble diagram offers a more in-depth look into one selected data flow showing the contents of the flows.

\subsection{Data Flow Graph}
\textbf{Simplified graphs.} When running an RP program Observables are created that depend on other Observables (their \textit{source}) and Observers are created to send their values to a defined set of Observers (their \textit{destination}). Figure \ref{chaincreate} shows these relations in a graph. For the simplest of programs the relations between the Observables ($O = {o_1, o_2, o_3}$) and those between Observers ($S = {s_1, s_2, s_3}$) share an equally shaped sub-graph after a reversal of the Observer-edges. To provide more overview we process the graph to merge the two Observable and Observer sequences together, simplifying it in the process, as in Figure \ref{fiddlesimple}. Higher order relations are retained as shown in \ref{fiddlehigher}.

\begin{figure}
	\centering
	\begin{tikzpicture}[->,>=stealth',shorten >=1pt,auto,node
    distance=1.5cm,
                    semithick,scale=0.8, every node/.style={scale=0.8}]
    \tikzstyle{every state}=[] \filldraw (0,2) circle (2pt) node[right]
    {of(1,2,3)} -- (0,1) circle (2pt) node[right] {map(\_ * 2)} -- (0,0)
    circle (2pt) node[right] {filter(\_ < 3)};
\end{tikzpicture}

	\caption{Simplified graph of Figure \ref{chaincreate}}
	\label{fiddlesimple}
\end{figure}

\begin{figure}
	\centering
	\begin{tikzpicture}[->,>=stealth',shorten >=1pt,auto,node
    distance=1.5cm,
                    semithick,scale=0.8, every node/.style={scale=0.8}]
    \tikzstyle{every state}=[] \filldraw (0,2) circle (2pt) node[right]
    {of(1,2,3)} -- (0,1) circle (2pt) node[right] {skip(1)} -- (0,0)
    circle (2pt) node[right] {flatMap(() => inner)};

    \filldraw (-1,1) circle (2pt) node[left] {of(`A', `B')} -- (0,0)
    circle (2pt) node[right] {};

    \filldraw (-0.7,1) circle (2pt) node[left] {} -- (0,0) circle (2pt)
    node[right] {};

\end{tikzpicture}

	\caption{Simplified graph of Figure \ref{chainhigher}}
	\label{fiddlehigher}
\end{figure}

\textbf{Layout} is used to add extra meaning to the graph. If multiple subscriptions on the same Observable are created, multiple flows are kept in the graph and they are layout together. This helps developers find related flows. Also it is easy to see when a Observable is reused many times, hinting a possible performance improvement by sharing the computation (Rx has special \code{share}-operators to multicast). The layout is based on StoryFlow~\cite{liu2013storyflow} which employs a hierarchical clustering before ordering the graph in a way to reduce crossings.

\textbf{Coloring} the nodes can be used to identify the same Observable in multiple places in the graph, as Observables can be reused in different places of the stream.

\subsection{Marble Diagram}
The term \textit{Marble Diagram} comes from the shape of the glyps in the images used to explain Rx in the official documentation. Different from the original diagrams we use dynamic diagrams which update live when new events occur and are stacked to show the data in the complete flow. Handcrafted marble diagrams can use custom shapes and colors to represent events, but for the generic debugger we use only three shapes: next events are a green dot, errors a black cross, completes a vertical line, as shown in Figure~\cite{marblediagram}.

 \begin{figure}
	\centering
	\begin{tikzpicture}[->,>=stealth',shorten >=1pt,auto,node distance=1.5cm,
                    semithick,scale=0.8, every node/.style={scale=0.8}]
\tikzstyle{every state}=[]
\tikzset{
  block/.style = {draw, minimum width=8cm, minimum height=0.6cm}
}

\node[minimum width=8cm, minimum height=0.6cm] (T) at (0cm, 0cm) {};
\node[below=0cm of T.west] (ULS) {};
\node[below=0cm of T.east] (ULE) {};
\node[block,below=0.3cm of T] (B) {\color{red}TODO replace with PDF/SVG diagram};
\node[below=0.5cm of B.west] (DLS) {};
\node[below=0.5cm of B.east] (DLE) {};

\draw[->] (ULS.west) -- (ULE.east);
\draw[->] (DLS.west) -- (DLE.east);

\end{tikzpicture}

	\caption{Marble Diagram}
	\label{marblediagram}
\end{figure} 

{\color{red}
\begin{verbatim}
visualization:
- 1. dependency graph
- 2. marble diagram

  motivation:
  - structure defined in code can be scattered
  - see high level structure (helps with composing)
  - see missing dependencies
  - see duplicate flow subscriptions (hurts performance)
  - zoom in to see internals of built in advanced op's
  - is basically automatic documentation

architecture of 3 components:
- 1. host instrumentation
- 2. analyzer: graphs from instrumentation output
- 3. visualizer

  motivation
  - independent instrumentation per platform
  - visualizer separated, so supports both 
    online (live) as offline (post mortem) viz
\end{verbatim}
}


{\color{red}
 The code relates to the setup phase, while the implementation of Rx leads to

RP in general and Rx in particular allow for data flows to be composited into new data flows. 


One can see that every composition from one or more input flows to one or more output flows can be visualized in the form of a graph, very node representing 
The visualizer focusses 
}

\subsection{Marble Diagrams}

\subsection{Implementation}
{\color{red}
\begin{verbatim}
5 stages:
- 1. instrumentation
- 2. linking / referencing / building a graph
- 3. sending the graph over from the client to the viz
- 4. building the graph on the other side
- 5. layout

explain independency of stages 3-5 on language & platform: 
can run with any Rx if stage 1 & 2 are implemented. 
Efforts done with RxJS 4, RxJS 5 and some limited 
work in RxJava.
\end{verbatim}
}

validation with test set of operators, to be discussed either here or at evaluation.

