\section{Motivation}
Understanding code is an important part of the daily life of a programmer. However, compared to programming in general, little is known about program comprehension and debugging practices for Reactive Programming. Existing knowledge about traditional imperative programs might not apply to RP. Many concepts of FP are borrowed by the RP libraries, about which Weck and Tichy say that ``higher-level abstractions like monads [...] can be a barrier for code comprehension"~\cite{weck2016visualizing}. Furthermore it is reiterated that there is a large gap between developers new to RP and experienced RP developers, at different occasions, including the recent Rx Contributor Days~\footnote{http://contributordays.com/contributor-days/rxjs}. To ``think in the reactive way'' requires comprehension of the core concepts that RP offers. As debugging is a fundamental part of comprehension, any improvements made to ease the debugging experience and capabilities in this area might improve how RP is comprehended.
The impact of any improvements on the developer community can be substantial, as recent reactive frontend frameworks like Angular and React grow in usage, and increasingly many developers are exposed to some form of RP. Doing something practically useful for this community is what lead us to create RxFiddle.

\section{RP Debugging practices}
\todo{alternative section titles: Evaluating motivation / collecting empirical evidence / case study}

To validate the need for better tools we must first understand what existing tools are used (RQ1).

\subsection{Literature and written work}
\begin{itemize}
	\item clues in books: recent book (2016) on RP by Esposito and Ciceri~\cite{esposito2016reactive}, has chapter `Debugging' which solely consists of ways to print to the console and how to transform streams (by writing code) that are more easily printed.
	\item article by Stalz on how to debug\footnote{http://staltz.com/how-to-debug-rxjs-code.html}: iterates that there are 3 main ways: drawing the dependency graph, drawing marble diagrams, tracing to the console.
	\item Official documentation\footnote{\url{https://github.com/Reactive-Extensions/RxJS/blob/master/doc/gettingstarted/testing.md\#debugging-your-rx-application}}: use \code{do}-operator to add tracing to console.
\end{itemize}

\subsection{Interviews}
\todo{
Performed interviews in a professional setting. Questions asked include `how do you debug Rx', `how do you test Rx'. Answers gathered, using cards method, categorized to concepts.
}

%\todo{problem, solution, why choices made, motivate the choices}
%As discussed in Section ref{background}, much has been written about comprehension \& debugging programs in traditional settings, but little is known specific to RP. The fact that tooling like debuggers were designed for imperative languages suggests that the current debuggers might not be optimal for RP.
%
%Salvaneschi et al.~\cite{salvaneschi2016debugging} report that students use . This suggestion is strengthened by the experience of the first author in a professional setting that developers struggle with debugging Rx, sprinkling the code with \code{println}-statements in order to gain insight in the runtime behavior, instead of using standard features like breakpoints and stepping.

%The research is motivated by multiple factors. 
%
% Initially, the first author experienced existing debugging practices in a professional setting.  
%  anecdotal experience
%  
%  
%debugging practices RP unknown;
%
%Salvaneshi; professional experience; RQ1
%
%
%how to visualize RP?



