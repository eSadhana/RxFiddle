\subsection{Reactive Programming}
\label{nutshell}

\epigraph{Reactive programs [..] maintain a continuous interaction with their environment, at a speed which is determined by the environment, not the program itself.}{Gérard Berry, 1989~\cite{berry1989real}}

Given this definition we can form some idea about what Reactive Programming is. If a program must interact with the environment at the speed of the environment it must be ready to handle input from the environment. Therefore programs need to declare how the input is handled and how to respond upfront. Reactive Programming languages and libraries provide developers with the appropriate abstractions and methods to create such programs.

The programming paradigm of Reactive Programming is implemented by multiple languages and libraries. Many RP implementations share a notion of a collection that abstracts over \textit{time}, in contrast to \textit{space} like standard collections, be it \textit{Observable} (Rx~\cite{meijer2010subject}), Signal (Elm~\cite{czaplicki2012elm}), Signal/Event (REScala~\cite{salvaneschi2014rescala}) or Behavior/Event (FRP~\cite{elliott1997functional}). In this paper we focus on the Rx formulation, but our work is applicable to other RP implementations to some extend. 

Understanding how we, given the Rx language, arrive at our visualization requires a minimal understanding of Rx.
Rx introduces two types \textit{Observable} and \textit{Observer}, derived by dualizing the Iterable and Iterator types. Observables define the data flow and Observers receive the data, possibly moving the data further down the stream. Figure \ref{sample1} shows a very basic example of a in situ data flow in Rx. First an Observable is created, here using the static \code{of} method, then dependent Observables are created using the \code{map} and \code{filter} methods on the Observable instance. Finally we \code{subscribe} to start the data flow and send the data in the flow to the console (eg. JavaScript's stdout).

\begin{figure}

\begin{subfigure}[a]{\columnwidth}
\inputminted[tabsize=2]{javascript}{listings/sample1.js}	
\caption{Rx code example}
\label{sample1}
\end{subfigure}

\begin{subfigure}[b]{\columnwidth}
\centering
\begin{tikzpicture}[->,>=stealth',shorten >=1pt,auto,node distance=2.8cm,
                    semithick,scale=0.8, every node/.style={scale=0.8}]
  \tikzstyle{every state}=[]

  \node[state] (A)                    {$o_1$};
  \node[state] (B) [right of=A]       {$o_2$};
  \node[state] (C) [right of=B]       {$o_3$};

  \node[state] (X) [below right =1cm of A] {$s_3$};
  \node[state] (Y) [right of=X]            {$s_2$};
  \node[state] (Z) [right of=Y]            {$s_1$};

  \path (B) edge   node {source}                   (A)
        (C) edge   node {source}                   (B)
        (A) edge [bend left] node {map(\_ * 2)}    (B)
        (B) edge [bend left] node {filter(\_ < 3)} (C)
        (X) edge   node[below] {destination}       (Y)
        (Y) edge   node[below] {destination}       (Z)
        (Z) edge   node {subscribe}                (C)
        (Y) edge   node {subscribe}                (B)
        (X) edge   node {subscribe}                (A);

  \draw[<-] (A) -- node[above] {of(1,2,3)} ++(-2cm,0);
  %\draw[<-] (Z) -- node[below] {subscribe} ++(2cm,0);

\end{tikzpicture}

\caption{Rx graph example}
\label{chaincreate}
\end{subfigure}

\caption{Creation and transform of Observables}

\end{figure}

It is important to note that Observables are lazy, they are the blueprint of a data flow. Only when the \code{subscribe} method of Observable is called the data flow is created by recursively subscribing up the stream. \textit{Observer}s are subscribed to each Observable until the source Observables are reached which then are setup and can start emitting the data.
This is illustrated in figure \ref{chaincreate}. Here $o_1$, $o_2$ and $o_3$ represent Observables defined in Figure \ref{sample1}. Inside the \code{subscribe} call $s_1$ is created and passed to $o_3$, which in turn will recursively subscribe to $o_2$ with a new Observer $s_2$ with destination $s_1$, until the full chain is subscribed.
The origin of this design is the duality between Observables and \textit{Iterables} - as first described by Erik Meijer, where Observers are dual to \textit{Iterators}.

\begin{figure}

\begin{subfigure}[a]{\columnwidth}
\inputminted[tabsize=2]{javascript}{listings/sample3.js}	
\caption{Higher order flatMap operation}
\label{sample3}
\end{subfigure}

\begin{subfigure}[b]{\columnwidth}
\centering
\begin{tikzpicture}[->,>=stealth',shorten >=1pt,auto,node distance=2.8cm,
                    semithick,scale=0.8, every node/.style={scale=0.8}]
  \tikzstyle{every state}=[]

  \node[state] (A)                    {$o_2$};
  \node[state] (B) [right of=A]       {$o_3$};
  \node[state] (C) [right of=B]       {$o_4$};

  \node[state] (X) [below right =1cm of A] {$s_3$};
  \node[state] (Y) [right of=X]            {$s_2$};
  \node[state] (Z) [right of=Y]            {$s_1$};
  
  \node[state] (F) [below =2cm of B]              {$o_1$};
  \node[state] (P) [below =1.8cm of Y] {$s_{4,n}$};
  \node[state] (Q) [below =0.3cm of P]       {$s_{4,m}$};

  \path (B) edge   node {source}                   (A)
        (C) edge   node {source}                   (B)
        (A) edge [bend left] node {skip(1)}    (B)
        (B) edge [bend left] node {flatMap(() => inner)} (C)
        (X) edge   node[below] {destination}       (Y)
        (Y) edge   node[below] {destination}       (Z)
        (Z) edge   node {subscribe}                (C)
        (Y) edge   node {subscribe}                (B)
        (X) edge   node {subscribe}                (A)
        
        (P) edge [bend right]  node[sloped,above] {subscribe}   (F)
        (P) edge   node[sloped,below] {destination} (Z)
        (Q) edge [bend left]  node[sloped,above] {subscribe}   (F)
        (Q) edge   node[sloped,below] {destination} (Z);
  
%  \node[draw=none] (K) at ($ (P) + (-4:-3) $){$\cdots$};
%  \node[draw=none] (M) at ($ (Q) + (-4:-3) $){$\cdots$};  
        
  % https://tex.stackexchange.com/questions/75497/automata-container-box
  \node [draw=red, label={\color{red}inner}, fit=(F) (P) (Q), inner sep=0.5cm,dashed, ultra thick, fill=red!10, fill opacity=0.2] {};
  %\node [draw=blue, fit= (G) (Q) (M), inner sep=0.5cm,dashed, ultra thick, fill=blue!10, fill opacity=0.2] {};

  % https://tex.stackexchange.com/questions/160643/ellipsis
  \draw[<-] (A) -- node[above] {of(1,2,3)} ++(-2cm,0);
  \draw[->] (F) -- node[above] {$\cdots$} ++(-2cm,0);
  \draw[<-] (P) -- node[above] {$\cdots$} ++(-3cm,0);
  \draw[<-] (Q) -- node[above] {$\cdots$} ++(-3cm,0);

\end{tikzpicture}

\caption{Higher order visualization}
\label{chainhigher}
\end{subfigure}

\caption{Higher order Observables}

\end{figure}

More complex programs feature operators that merge Observables, split Observables or handles higher order Observables, resulting in more complex graphs. While merging and splitting happens on an Observable level (the \code{source} property still points to the dependency) higher order Observable flattening manifests only in Observer structure, as shown in Figure \ref{chainhigher}.

Creating the Observable we will call the \textit{assembly} phase, the phase where the subscribe happens the \textit{subscription} phase and data flows in the \textit{runtime} phase. The three phases can be interleaved for different streams, for example when dealing with higher order Observables,  meaning one could use Observables as values inside the data flow. The Observables used as values have yet to start the second phase while the outer stream is in the runtime phase.
