\section{Study Design}
\todo{
qualitive data
quantitive
design, implemented, tested with real developers
buzz words: mixed methods, grounded theory

Design:
- split Marble Diagram to nutshell and updated methods
}
\section{RP Debugging practices}
\todo{alternative section titles: Evaluating motivation / collecting empirical evidence / case study}

To validate the need for better tools we must first understand what existing tools are used (RQ1).

\subsection{Literature and written work}
\todo{
\begin{itemize}
	\item clues in books: recent book (2016) on RP by Esposito and Ciceri~\cite{esposito2016reactive}, has chapter `Debugging' which solely consists of ways to print to the console and how to transform streams (by writing code) that are more easily printed.
	\item article by Stalz on how to debug\footnote{http://staltz.com/how-to-debug-rxjs-code.html}: iterates that there are 3 main ways: drawing the dependency graph, drawing marble diagrams, tracing to the console.
	\item Official documentation\footnote{\url{https://github.com/Reactive-Extensions/RxJS/blob/master/doc/gettingstarted/testing.md\#debugging-your-rx-application}}: use \code{do}-operator to add tracing to console.
\end{itemize}
}

\subsection{Interviews}
\todo{
Performed interviews in a professional setting. Questions asked include `how do you debug Rx', `how do you test Rx'. Answers gathered, using cards method, categorized to concepts.
}

%\todo{problem, solution, why choices made, motivate the choices}
%As discussed in Section ref{background}, much has been written about comprehension \& debugging programs in traditional settings, but little is known specific to RP. The fact that tooling like debuggers were designed for imperative languages suggests that the current debuggers might not be optimal for RP.
%
%Salvaneschi et al.~\cite{salvaneschi2016debugging} report that students use . This suggestion is strengthened by the experience of the first author in a professional setting that developers struggle with debugging Rx, sprinkling the code with \code{println}-statements in order to gain insight in the runtime behavior, instead of using standard features like breakpoints and stepping.

%The research is motivated by multiple factors. 
%
% Initially, the first author experienced existing debugging practices in a professional setting.  
%  anecdotal experience
%  
%  
%debugging practices RP unknown;
%
%Salvaneshi; professional experience; RQ1
%
%
%how to visualize RP?



